\section{Implementation \& Design}
  For this paper Fibonacci and the Sort algorithm are chosen to be implemented from BOTS, as introduced in subsection \ref{subsec:BOTS}.
  Fibon

	- Fibonacci --> high amount of tasks, having the scheduling as the main part of the workload
	- MergeSort --> easy to implement, good use case and high number of tasks


- own implementation
  - small / medium / large task sizes
  - dependencies
  - can vary the task size
  
- from BOTS
	as already mentioned in the section before
	-> Fibonacci
	-> MergeSort
		
		
\subsection{Optimization}
	
	\cite{LaGrone.2011}
	--> task has an if clause evaluated to false will directly continue with the generated task --> depth-first
	--> untied tasks (balancing) vs tied (data locality)
	--> (maybe) use taskwait for less scheduling overhead --> depth-first
	--> cutoff, regulating task creation, if condition is not met execute tasks immediately and do not place them on the queue
		--> create new OMP\_TASK\_CREATE\_COND environment variable
		
    --> does the shared variable directive decrease the work for task creation???
    
    --> a nowait directive at the omp single
    
    \cite{MKlemm.2018}
 --> directives to optimize
 	-->taskyield - suspend the current task in favor of execution of a different task
 	--> if - in case it is false the task is executed immediately
 	--> mergeable - A task for which the data environment, is the same as that of its generating task region
 	--> final - force all child tasks to become final and included
 		--> if true --> child tasks become included --> means that the tasks are also executed by the parent task


  \cite{TheSTEARGroup.15.01.2020}
    - HPX --> Performance counters to identify bottlenecks
    - HPX exposes special API functions to allow one to create, manage and read the counter data
    - all performance counter instances have a unique name to access the counter