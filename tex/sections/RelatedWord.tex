\section{Related Work}

\subsection{The Task Feature}
 2 - important for applications exhibiting irregular parallelism (recursion, pointer-chasing)
 2 - downside: increased complexity --> poor implementations can lead to to overhead
 2 - performance of an application heavily relies on the scheduler
 2 - ideal: maximize concurrency, load balancing and locality
 	-> efficiency depends on data structures for storing unfinished tasks, manage task switching, regulation of task creation, task synchronization and task memory footprint
 2 - schedulers mostly have queues for storing tasks that are ready for execution
 	--> several ways of queue organization (there is a source to cite)
 		- 1 queue n threads / n threads n queues / hierarchical tree structure ...
 2 - schedulers can be classified in depth- and breadth-first 
 	depth --> create child task and execute it --> less number of tasks leading to less concurrency opportunities, however better data reuse
 	

\subsection{OpenMP}

explain OpenMP ? -- basic knowledge ?

1\cite{Ayguade.2009}
2\cite{LaGrone.2011} --> some further information and alternatives can be found here
3\cite{MKlemm.2018}
4\cite{Qawasmeh.2014}
5\cite{Duran.2008}
6\cite{Furlinger.2009}

example implementation here - show the directives

  1,2- openMP shifts to tasks in version 3.0
  1,2- introduces task directive, to specify explicit tasks
  	1,2- explicit -> defined by programmer
    2,3- implicit tasks are created in parallel regions 
    --> the program does not need to specify or know about them
  1,2- explicit tasks can be executed by any thread in the current team
  1- are defined by the task construct and enclose a certain code area defined by the programmer --> task region
  1- like in normal parallel regions variable scopes can be defined by shared, private or firstprivate
  1,2- by default, a thread is tied to a task as it begins to execute the task
    1,2- this means that the task is only executed by this thread, however the thread can still execute other tasks in case the task reaches a task scheduling point and has to suspend its work
    1,2- this restriction can be avoided by defining the tasks untied
    
  5- when a thread encounters a task directive, the data environment is captured
  		--> the environment and the code block constitute the generated task


  4- OpenMP runtime becomes responsible for the scheduling of tasks
  4- task construct allows developers to dynamically create asynchronous units of work
  2- explicit scheduling can be achieved by 
      - \texttt{taskwait}, suspend encountering task region and wait for all child tasks to complete
      - \texttt{barrier}, wait till all reach the barrier

  6- tasks result in more dynamic and unpredictable execution characteristics
  	--> does the thread switch to the child task or not ? --> use true false at the creation to minimize that uncertainty


\subsection{HPX}
\subsubsection{HPX}
  \cite{Kaiser.2014}
  - general purpose C++ runtime system for parallel and distributed applications of any scale
  - innovative mixture of global system-wide address space, fine grain parallelism, and lightweight synchronization
    + implicit message driven computation
    + semantic equivalent of local and remote execution
    + explicit support for accelerators such as GPGPUs
  - aims to resolve scalability / resiliency / power efficiency / runtime adaptive resource management in the evolve from Peta- to Exascale systems
  
\cite{Kaiser.2009}
  - introduce a new model of parallel execution --> Parallex
    - efficiency / scalability
    
  - HPX = run time system supporting the ParalleX model
  
--> folgende sind große Vergleiche zu MPI ---
    - dynamically schedule multiple threads moving the work to the data
     -> instead of statically designated processes synchroized by message passing
    - use local synchronization and global asynchronicity (MPI uses messages as synchronization)
    - Global barriers are essentially eliminated and instead replaced by
lightweight Local Control Objects (LCOs) -> futures / mutexes
    - key to efficiency and latency hiding is the message-driven work-queue
	  -> apply user tasks to physical processing resources
	- uses active global address space (AGAS)
	  - allows to move virtual objects in physical space without changing virtual names
	  ---
	- conventiionally 1 process = 1 processor
	  -> ParalleX, use \textit{parallel processes}: map a process to multiple cores, allow many concurrent threads and child processes
	- use a thread manager implemented as FCFS scheduler 
	  - OS threads work from a single queue of tasks 
	  --> sufficient for small amount of concurrent OS threads
	  --> there is work on a more scalable , work stealing scheduler using one queue per OS thread (core)
	  

\subsubsection{Tasks in HPX}


\subsection{hpxMP}
\cite{TianyiZhang.2019}
- introduction of hpxMP

\cite{Zhang.2192020}
- OpenMP 5 tasks in hpxMP


    
\subsection{Benchmark Algorithm}
- Barcelona OpenMP Task Suite (BOTS)
	- aim of the suite is to provide a collection of benchmarks that would allow vendors to test the impact of different implementation decisions in a multicore architecture
	- 9 benchmarks
		> Alignment: align protein sequences against every other sequence using a special algorithm
		> FFT: Fast Fourier Transformation
		> Fibonacci: calculation of the nth number in the Fibonacci sequence
		> Floorplan: compute the optimal floorplan distribution of a number of cells
		> Health: simulate the Columbian Health Care System
		> N Queens: the n-queen problem, to find placements of queens on a chessboard under special conditions
		> Sort: parallel execution of merge sort
		> SparseLU: LU matrix factorization over sparse matrices
		> Strassen: hierarchical decomposition of a matrix for multiplication of large dense matrices
	--> different version of each benchmark and some with cut-off version to avoid high amount of tasks
	

	- Fibonacci --> high amount of tasks, having the scheduling as the main part of the workload
	- MergeSort --> easy to implement, good use case and high number of tasks
	
