\section{Introduction}
  Task-based programming increased in popularity in recent years.
  It allows to parallelize irregular and dynamic structures like while loops, dynamic linked list traversals and recursive function calls.~\cite{Ayguade.2009}
  Open Multi-Processing (OpenMP) and High Performance ParalleX (HPX) are both parallel run time system which increased focus on task-based parallelism.~\cite{TheSTEARGroup.2020}
  OpenMP for example as de facto standard for shared-memory parallel programming did increase its directive pool concerning tasks in the latest version, 5.0.~\cite{Zhang.2192020}
  These two run time systems have not yet been compared in performance of task-based parallel execution.
  This paper tries till fill this gap.
  
  The main focus is put on the task schedulers of both systems.
  Task schedulers play a crucial role as they have to manage load balancing and data locality.
  Having the same computation workload the differences in performance will emerge from these components.~\cite{Qawasmeh.2014}
  
  Section \ref{sec:RelWork} will introduce the tasking principles and shows how they are used in both run time systems.
  It furthermore explains a new approach about using OpenMP tasks in HPX.
  Then a short overview of the experiment environment is granted and section \ref{sec:implem} presents the used benchmark algorithms to compare OpenMP and HPX.
  Afterwards some further experiments are conducted in section \ref{sec:Opt}.
  Optimization options are tested and evaluated before a conclusion is drawn in the final section.