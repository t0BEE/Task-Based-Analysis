\section{Introduction}
  Task-based programming increased in popularity in recent years.
  It allows to parallelize irregular and dynamic structures like while loops, dynamic linked list traversals and recursive function calls.~\cite{Ayguade.2009}
  OpenMP and HPX are both parallel run time system which increased focus on task-based parallelism.~\cite{TheSTEARGroup.2020}
  OpenMP as de facto standard for shared-memory parallel programming did even increase its directive pool concerning tasks in the latest version, 5.0.~\cite{Zhang.2192020}
  These two run time systems have not yet been compared in performance of task-based parallel execution.
  This paper tries till fill this gap.
  The significant difference in performance will be due to the task scheduler of both systems as the workload for execution has to be equal.
  Therefore the task scheduler plays a crucial role as it has to manage load balancing and data locality.~\cite{Qawasmeh.2014}
  
  Section \ref{sec:RelWork} will introduce the tasking principles and shows how they are used in both run time systems.
  It furthermore explains a new approach about using OpenMP tasks in HPX.
  Afterwards section \ref{sec:implem} presents the used benchmark algorithm to compare OpenMP and HPX.
  Then a short overview of the experiment environment is granted, followed by the actual experiments in section \ref{sec:Opt}.
  The two run time systems are compared to each other and a sequential implementation.
  Further optimization options are tested and evaluated at the end before a conclusion is drawn.